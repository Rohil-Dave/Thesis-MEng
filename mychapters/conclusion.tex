\chapter{Conclusion}

In this project, a method for parameterizing a zero-waste pattern for a short-sleeved collared shirt was developed to address fabric waste and optimize garment fit. The parameterized pattern provides end users with a choice of designs: a true zero-waste design that is fully customized and another design that minimizes waste as much as possible for an ideal bolt width. This approach acknowledges the traditional fabric waste in conventional pattern-making processes, suggests fabric cut loss utility strategies and offers practical insights for balancing fit and sustainability. The outputs of the project not only create customizable patterns but also render files that can be imported into CLO for virtual draping and visualization. This provides users with a comprehensive view of the final garment before physical production, enhancing their ability to make informed decisions about fit and fabric utilization. Workshop findings have provided proof of concept for the parameterization approach, demonstrating its feasibility and practical application. However, the fit findings need more extensive testing to be generalized further. Further physical testing should involve a larger and more diverse sample size to validate the fit and ensure the pattern's applicability across various body types. Analysis on a publicly available dataset of scanned bodies verified pattern suitability across efficiency metrics, showing that custom size patterns using an ideal bolt have at least 95\% fabric use efficiency. Future enhancements could include expanding the pattern parameterisation options to accommodate a wider range of customisation to improve fit accuracy and to integrate the framework into a webtool to streamline the user experience. It is feasible to apply the methods of this framework to parameterise different pattern designs. This project represents a significant step towards sustainable fashion practices by providing valuable insights and practical tools for independent designers. By offering customisable patterns with clear metrics on waste efficiency, the project empowers users to make informed choices, balancing fit and sustainability in their garment production.