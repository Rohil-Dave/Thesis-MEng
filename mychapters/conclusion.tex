\chapter{Conclusion}

This project parameterised a zero-waste pattern of a short-sleeved collared shirt based on user body measurements. It addressed fabric waste by fitting the parameterisation to an ideal bolt width. Techniques were developed for segmentation and reconstruction as well as recoupment of cut losses via embellishments when other bolt widths were used by consumers.

The outputs of the project were patterns and instructions directed at confident beginners and DXF files that can be imported into CLO for virtual draping. This enables a visual representation of the final garment before physical production, allowing them to gauge fit and fabric utilisation.
Workshop findings proved the concept, demonstrating its feasibility and practical application. Further physical testing of a larger and more diverse sample size is needed to validate the fit and viability across various body types. Analysis on a publicly available dataset provided efficiency metrics (98.1\% mean) for ideal bolt widths.

Future enhancements could include streamlining the framework with better UX design, more customization of the pattern for a better fit, and parameterising different pattern designs. 
This project is a first step towards sustainable fashion practices providing practical tools and clear efficiency metrics to independent fashion designers. It empowers them to make informed choices, balancing fit and sustainability in their garment production.
