\chapter{Discussion}

\section{Findings}
\subsubsection{Precisions}
All body measurements used in the workshop were rounded up to 0.5cm ceilings. While this does affect the fit (slightly looser) and efficiency (minimally higher) it does not affect the analysis that much. The reason for this is that this pattern is not machine cut. The target audience is confident beginners and not industrial tailors. Therefore, precise millimetre level cutting was not a fair expectation.

\section{Limitations}

\subsection{Method Limitations}
pattern generalised

constants: collar, sleevehead curve, neck opening

agressive scaling factors:

neck circ should influence neck face

design should be dynamic, increase ease for circs greater than 100 cm

manual measure may have error

scanning tech not as accessible

sewing inconsistency

A limitation of this workshop was that there was only one bolt of calico fabric (150cm) available. This means that the participants could not all get their ‘ideal’ bolt widths for true zero waste design. This will show up in lower efficiency scores and increased cut loss.

\subsection{Data Limitations}
self-reported data, which can be subject to biases and inaccuracies. subjective assessments of fit and comfort may vary based on personal preferences and experiences. 

controlled conditions for waste

Limited Data, generalising needs more test

only 8 participants only 6 finished

time resource constraints

\subsection{tail0r Limitations}
Limitations with CLO

Posture mode is quite static in CLO and does not allow for varied postures

So we are simulating the fit in many different natural poses

Does not give much more than a static visual representation of fit

\section{Outcome Impact and Potential}

\section{Further Work}
parameter tuning, neck, shoulder etc

building in fabric properties

building in costing?

automation of pipeline

parameterise other patterns