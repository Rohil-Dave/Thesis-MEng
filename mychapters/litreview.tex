\chapter{Literature Review}
This literature review aims to explore fabric waste at the point of garment manufacturing, examining traditional pattern making processes, the advent and principles of zero waste design, the methods and rationale behind bespoke clothing, the concept of garment utility, and the impact of digitisation trends on these processes. By analysing these aspects, the review seeks to provide a comprehensive understanding of the current challenges and innovations in reducing fabric waste and how solutions that promote sustainable practice in the fashion industry can be formed. 

Example citation \cite{aldrich_metric_2015}

\section{Fabric Waste at the Point of Garment Manufacturing}
addd this bro

In the context of minimizing fabric waste or recouping fabric waste, designers should consider how they might provide utility to the fabric cut loss to increase the garment’s overall utility. Garment utility refers to the functionality, durability, and versatility of a garment, as well as its aesthetic features that are desirable to the user. Enhancing garment utility can also involve incorporating fabric cut losses into the garment’s design in creative ways, thereby reducing waste and increasing the garment’s overall appeal and lifespan.


\section{Traditional Pattern Making}
Traditional pattern making is a fundamental aspect of garment production, involving the creation of templates for cutting fabric pieces. This process includes several critical steps:

Traditional pattern making begins with creating a basic pattern block based on precise body measurements. This block serves as a foundation for various garment styles. Designers then modify the block to create different garment designs through techniques such as dart manipulation and adding fullness. Once the basic pattern is established, grading is used to scale the pattern to different sizes. This involves adding or subtracting specific amounts at strategic points to maintain the proportions of the original design. The increments that define sizing are typically derived from grade rules that calculate half the difference between the next size measurement and the current size measurement.

However, traditional grading systems often fail to accommodate diverse body shapes and sizes, thereby limiting inclusivity in fashion. For example, there are ‘big/tall’ and ‘petite’ sizes, as well as size nomenclature that lacks specific body measurements (e.g., ‘women’s size 6’). Additionally, manufacturers often do not produce clothing in all sizes, further limiting consumer choice. This practice is part of their ‘branding’ strategy, targeting their offerings to a specific subset of body types. Standardized sizing systems are based on median body measurements within a target demographic, with guidelines like ISO 8559 ensuring consistency across manufacturers and markets.

Traditional pattern pieces placed on a marker are cut out separately, surrounded by negative space. This separation often results in pieces not sharing borders, leading to unusable offcuts. While effective for mass production, traditional methods often result in significant fabric waste. Fletcher comments that the current approach and mindset accept these losses as an inevitable and acceptable part of the supply chain. Patterns are not always optimized for fabric efficiency, and the rigidity of traditional pattern making limits creativity and innovation, making it difficult to adapt to new sustainable practices, particularly on industrial scales.


\section{Advent and Principles of Zero Waste Design Practices}

\section{Bespoke Clothing}

\section{Digitisation}


\section{Functionality}

\subsection{Garment Utility}
The concept of garment utility is particularly relevant in the context of sustainable fashion. Beyond reducing waste in the manufacturing process, the fashion industry should focus on making clothing with increased usability. Garment utility doesn’t just encompass functionality, but also durability, versatility, and aesthetic value. High-utility garments are highly valued by consumers as they are designed to be adaptable to various occasions, reducing the need for multiple items and thus decreasing overall consumption. This is a core principle of ‘slow fashion’, which advocates for mindful consumption and long-term use of clothing. By focusing on garment utility, designers can create more sustainable fashion products that meet the needs of consumers while minimising environmental impact.

\subsection{Traditional Pattern Making}
Traditional pattern making involves creating a two-dimensional template for garment pieces, which are then cut and assembled. These templatized patterns are adjusted based on standardised sizing charts in a process called ‘grading’. Traditional pattern making often results in significant fabric waste, as patterns are not always optimised for fabric efficiency. 
Grading has also been criticised for its lack of inclusivity, as it often does not accommodate diverse body shapes and sizes. Examples abound of ‘big/tall’ sizes, ‘petite’ sizes, size nomenclature devoid of body measurements (e.g. ‘women’s size 6). Often, manufacturers do not make clothing in all sizes limiting consumer choice. This is also part of their ‘branding’ exercise, targeting their offerings to a certain subset of body types. The rigidity of traditional pattern making limits creativity and innovation, making it difficult to adapt to new sustainable practices.


\section{Innovation}

\subsection{Digitisation Trends in Fashion}
The digitization of fashion processes has revolutionised the industry. Virtual fitting and draping technologies are becoming increasingly popular, allowing designers and consumers to visualise garments on digital avatars before physical production. These technologies reduce the need for physical samples, thus saving resources and minimising waste. Virtual reality and augmented reality are also being used to create immersive shopping experiences, further integrating technology into the fashion landscape. Additionally, the data collected from these and other sources allows manufacturers to better predict demand and right size their production quantities. These advancements improve the efficiency of the design process and also enhance the overall consumer experience.

\subsection{Computational Frameworks for Garment Creation}
The development of computational frameworks for garment creation is transforming the fashion industry. Tools such as GarmentCode and 3D Look enable designers to create digital patterns, streamline the design process, reduce the need for physical samples, and enhance the accuracy of garment construction.
These frameworks allow for the optimization of fabric use,  reduction of waste and ultimately contribute to more sustainable fashion practices. They also bridge the gap between industry standards and independent/domestic domains, making advanced design techniques accessible to smaller designers and home sewers.


\subsection{Bespoke Clothing}
Traditional sizing is a significant barrier to inclusivity in the fashion industry. Standard sizing charts fail to accommodate the diverse range of body shapes and sizes, leading to ill-fitting garments and customer dissatisfaction. Bespoke clothing offers a solution to this problem. Heretofore, the bespoke tailoring was a manual, labour intensive, costly, and time-consuming process. Recent advancements in body scanning technology offer a promising alternative. 3D body scanning enables precise measurement taking, ensuring better fit and reducing the likelihood of returns and exchanges. These technologies enhance the inclusivity of fashion by allowing for a wider range of sizes and fits, making it possible to create garments that cater to diverse body types.


\section{Zero Waste Pattern Design}

\subsection{Pre-existing Limitations}
Current methods to address fabric waste in garment manufacturing often focus on the later stages of production. Techniques such as pattern engineering aim to modify garment patterns post-design to improve fabric yield, but these are limited by the initial design constraints. The marker-making process, whether manual or computer-aided, attempts to arrange pattern pieces efficiently on the fabric, but is inherently restricted by the predetermined pattern shapes. Therefore, significant fabric waste persists despite these efforts.

\subsection{Paradigm Shift}
Zero-waste pattern design represents a paradigm shift in addressing fabric waste. The history of zero-waste design can be traced back to traditional garment-making practices in various cultures, where fabric was a precious resource. The context provided by researchers like Burnham (1973) and the innovative work of contemporary designers like Rissanen and McQuillan demonstrate the feasibility and aesthetic potential of zero-waste fashion. Unlike traditional methods, zero-waste design integrates waste minimization from the outset, creating patterns that utilise every square inch of fabric. This approach not only eliminates offcuts but also challenges designers to rethink garment construction creatively.

\subsection{Creative Imagination}
Key techniques in zero-waste design include the use of simple shapes, modularity, symmetry, and tessellation. The entire pattern typically fits into a compact rectangle. This leaves only small pieces of off cuts as the bolt widths come in standardised sizes. Designers further minimise waste by repurposing excess ‘retakes’ as embellishments and ‘mendings’. The goal is to find the ideal bolt width given the pattern dimensions to reduce cut losses and to the increase cut loss utility via creative re-use.
The unique challenges of zero-waste design require a high level of creativity and problem-solving skills from designers, as they must rethink traditional garment construction techniques. Creative imagination is crucial in navigating the constraints of fabric width and pattern interlocking to create functional and aesthetically pleasing garments.

\subsection{Design Ease Aesthetics}
Ease is the difference between the body measurements and the final garment measurements. The extra space in the garment that allows the user to move, sit and breathe while wearing the garment. Since zero waste patterns use simple shapes and modular construction, these garments have more ease and a looser fit. This alleviates the standardised grading problems of traditional patterns as the finished garments are not form fitting.

\subsection{Application in Independent and Domestic Domains}
While large-scale industry production often overlooks fabric efficiency due to economic and logistical constraints, independent fashion designers and domestic sewers are well-positioned to adopt zero-waste techniques. These smaller-scale operations can experiment with and implement zero-waste designs more flexibly, contributing to a reduction in fabric waste at a grassroots level. By integrating zero-waste principles into their practices, independent designers and home sewers can create unique, sustainable garments that challenge the norms of traditional fashion design.


