\chapter{Literature Review}
Summarize the breakdown of the lit review here to present the logical flow. Dont make the reader guess the flow before reading the lit review.

\section{Sustainability Crisis in the Fashion Industry} \label{sec:sections}

\subsection{Sourcing}
The environmental impact of the fashion industry is immense. The problem starts from yarn production whether harvesting natural materials like cotton or the production of synthetic fibres. For example, cotton farming relies heavily on pesticides and fertilisers, which contribute to soil and water pollution. The Aral Sea in Central Asia has experienced significant shrinkage due to the diversion of water for cotton production, leading to ecological disaster (Black, 2012). Polyester production releases greenhouse gases into the atmosphere and microplastics into waterways, which are ingested by marine life and enter the food chain, posing risks to human health (Napper \& Thompson, 2016). The dyeing and finishing processes in textile manufacturing release harmful chemicals into water bodies, affecting aquatic ecosystems and human health.

\subsection{Overproduction and Overconsumption}
The fast fashion business model results in a relentless cycle of overproduction and overconsumption. Brands rapidly produce large quantities of low-cost and low quality garments to chase ever-changing trends and induce consumer demand. The social pressure on consumers to frequently update their wardrobe encourages a throwaway culture, where clothing is discarded after only a few uses. Pan (2014) describes this as a ‘wicked problem’ characterised by short product life cycles and excessive waste. Estimates suggest that by 2050, the number of garments produced annually will reach a staggering 814 billion[citation needed], with upwards of 9.5 trillion garments to be produced from now till then. Since the demand for a particular style is hard to predict, brands overproduce garments. This not only strains natural resources but also leads to unsold inventory, which is often incinerated or sent to landfills.

\subsection{Textile Waste}
The traditional pattern-cutting process is inherently wasteful, with approximately 15\% of fabric discarded during the cutting process (Black, 2012). Patterns are often designed without considering fabric efficiency, leading to significant offcuts. Even with advanced software and skilled manual marker-makers, fabric wastage remains between 10\% to 20\% for adult outerwear (Feyerabend, 2004). Fletcher (2014) critiques the fact that waste minimization is not integrated into the design phase but the industry accepts these losses as an inevitable part of the supply chain. 
Additionally, the disposal of post-consumer textile waste exacerbates the problem. The accumulation of textile waste in landfills releases methane and leaches toxic substances into the soil and groundwater. Recycling efforts are often hampered by the complexity of separating blended fibres and the lack of efficient recycling technologies. The industry must address these inefficiencies to reduce its environmental impact.


\section{Functionality} \label{sec:sections}

\subsection{Garment Utility}
The concept of garment utility is particularly relevant in the context of sustainable fashion. Beyond reducing waste in the manufacturing process, the fashion industry should focus on making clothing with increased usability. Garment utility doesn’t just encompass functionality, but also durability, versatility, and aesthetic value. High-utility garments are highly valued by consumers as they are designed to be adaptable to various occasions, reducing the need for multiple items and thus decreasing overall consumption. This is a core principle of ‘slow fashion’, which advocates for mindful consumption and long-term use of clothing. By focusing on garment utility, designers can create more sustainable fashion products that meet the needs of consumers while minimising environmental impact.

\subsection{Traditional Pattern Making}
Traditional pattern making involves creating a two-dimensional template for garment pieces, which are then cut and assembled. These templatized patterns are adjusted based on standardised sizing charts in a process called ‘grading’. Traditional pattern making often results in significant fabric waste, as patterns are not always optimised for fabric efficiency. 
Grading has also been criticised for its lack of inclusivity, as it often does not accommodate diverse body shapes and sizes. Examples abound of ‘big/tall’ sizes, ‘petite’ sizes, size nomenclature devoid of body measurements (e.g. ‘women’s size 6). Often, manufacturers do not make clothing in all sizes limiting consumer choice. This is also part of their ‘branding’ exercise, targeting their offerings to a certain subset of body types. The rigidity of traditional pattern making limits creativity and innovation, making it difficult to adapt to new sustainable practices.


\section{Innovation} \label{sec:sections}

\subsection{Digitisation Trends in Fashion}
The digitization of fashion processes has revolutionised the industry. Virtual fitting and draping technologies are becoming increasingly popular, allowing designers and consumers to visualise garments on digital avatars before physical production. These technologies reduce the need for physical samples, thus saving resources and minimising waste. Virtual reality and augmented reality are also being used to create immersive shopping experiences, further integrating technology into the fashion landscape. Additionally, the data collected from these and other sources allows manufacturers to better predict demand and right size their production quantities. These advancements improve the efficiency of the design process and also enhance the overall consumer experience.

\subsection{Computational Frameworks for Garment Creation}
The development of computational frameworks for garment creation is transforming the fashion industry. Tools such as GarmentCode and 3D Look enable designers to create digital patterns, streamline the design process, reduce the need for physical samples, and enhance the accuracy of garment construction.
These frameworks allow for the optimization of fabric use,  reduction of waste and ultimately contribute to more sustainable fashion practices. They also bridge the gap between industry standards and independent/domestic domains, making advanced design techniques accessible to smaller designers and home sewers.


\subsection{Bespoke Clothing}
Traditional sizing is a significant barrier to inclusivity in the fashion industry. Standard sizing charts fail to accommodate the diverse range of body shapes and sizes, leading to ill-fitting garments and customer dissatisfaction. Bespoke clothing offers a solution to this problem. Heretofore, the bespoke tailoring was a manual, labour intensive, costly, and time-consuming process. Recent advancements in body scanning technology offer a promising alternative. 3D body scanning enables precise measurement taking, ensuring better fit and reducing the likelihood of returns and exchanges. These technologies enhance the inclusivity of fashion by allowing for a wider range of sizes and fits, making it possible to create garments that cater to diverse body types.


\section{Zero Waste Pattern Design}

\subsection{Pre-existing Limitations}
Current methods to address fabric waste in garment manufacturing often focus on the later stages of production. Techniques such as pattern engineering aim to modify garment patterns post-design to improve fabric yield, but these are limited by the initial design constraints. The marker-making process, whether manual or computer-aided, attempts to arrange pattern pieces efficiently on the fabric, but is inherently restricted by the predetermined pattern shapes. Therefore, significant fabric waste persists despite these efforts.

\subsection{Paradigm Shift}
Zero-waste pattern design represents a paradigm shift in addressing fabric waste. The history of zero-waste design can be traced back to traditional garment-making practices in various cultures, where fabric was a precious resource. The context provided by researchers like Burnham (1973) and the innovative work of contemporary designers like Rissanen and McQuillan demonstrate the feasibility and aesthetic potential of zero-waste fashion. Unlike traditional methods, zero-waste design integrates waste minimization from the outset, creating patterns that utilise every square inch of fabric. This approach not only eliminates offcuts but also challenges designers to rethink garment construction creatively.

\subsection{Creative Imagination}
Key techniques in zero-waste design include the use of simple shapes, modularity, symmetry, and tessellation. The entire pattern typically fits into a compact rectangle. This leaves only small pieces of off cuts as the bolt widths come in standardised sizes. Designers further minimise waste by repurposing excess ‘retakes’ as embellishments and ‘mendings’. The goal is to find the ideal bolt width given the pattern dimensions to reduce cut losses and to the increase cut loss utility via creative re-use.
The unique challenges of zero-waste design require a high level of creativity and problem-solving skills from designers, as they must rethink traditional garment construction techniques. Creative imagination is crucial in navigating the constraints of fabric width and pattern interlocking to create functional and aesthetically pleasing garments.

\subsection{Design Ease Aesthetics}
Ease is the difference between the body measurements and the final garment measurements. The extra space in the garment that allows the user to move, sit and breathe while wearing the garment. Since zero waste patterns use simple shapes and modular construction, these garments have more ease and a looser fit. This alleviates the standardised grading problems of traditional patterns as the finished garments are not form fitting.

\subsection{Application in Independent and Domestic Domains}
While large-scale industry production often overlooks fabric efficiency due to economic and logistical constraints, independent fashion designers and domestic sewers are well-positioned to adopt zero-waste techniques. These smaller-scale operations can experiment with and implement zero-waste designs more flexibly, contributing to a reduction in fabric waste at a grassroots level. By integrating zero-waste principles into their practices, independent designers and home sewers can create unique, sustainable garments that challenge the norms of traditional fashion design.


