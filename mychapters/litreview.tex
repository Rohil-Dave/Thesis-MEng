\chapter{Literature Review}
This literature review aims to explore fabric waste at the point of garment manufacturing, examining traditional pattern making processes, the advent and principles of zero waste design, the methods and rationale behind bespoke clothing, the concept of garment utility, and the impact of digitisation trends on these processes. By analysing these aspects, the review seeks to provide a comprehensive understanding of the current challenges and innovations in reducing fabric waste and how solutions that promote sustainable practice in the fashion industry can be formed. 

Example citation \cite{aldrich_metric_2015}

\section{Fabric Waste During Garment Manufacturing}
\textbf{This 1st para is bad, has no substance}

Fabric is a significant issue in the fashion industry, with an estimated 15-20\% of fabric wasted during garment production. This waste occurs at various stages of the manufacturing process, including cutting, sewing, and finishing. The cutting stage is particularly problematic, as fabric cut losses can account for up to 30\% of the total fabric used. This inefficiency not only contributes to environmental degradation but also increases production costs and reduces profit margins for manufacturers. Addressing fabric waste at the point of garment manufacturing is crucial for promoting sustainability in the fashion industry and reducing its environmental impact.

In the context of minimizing fabric waste or recouping fabric waste, designers should consider garment utility. Garment utility refers to the functionality, durability, and versatility of a garment, as well as its aesthetic features that are desirable to the user. Enhancing garment utility can also involve incorporating fabric cut losses into the garment’s design in creative ways, simultaneously reducing waste and increasing the garment’s overall appeal and lifespan. This is a core principle of ‘slow fashion’, which advocates for mindful consumption and long-term use of clothing. By focusing on garment utility, designers can create more sustainable fashion products that meet the needs of consumers while minimising environmental impact.

\section{Traditional Pattern Making}
Traditional pattern making is a fundamental aspect of garment production, involving the creation of templates for cutting fabric pieces. This process includes several critical steps:

Traditional pattern making begins with creating a basic pattern block based on precise body measurements. This block serves as a foundation for various garment styles. Designers then modify the block to create different garment designs through techniques such as dart manipulation and adding fullness. Once the basic pattern is established, grading is used to scale the pattern to different sizes. This involves adding or subtracting specific amounts at strategic points to maintain the proportions of the original design. The increments that define sizing are typically derived from grade rules that calculate half the difference between the next size measurement and the current size measurement.

However, traditional grading systems often fail to accommodate diverse body shapes and sizes, thereby limiting inclusivity in fashion. For example, there are ‘big/tall’ and ‘petite’ sizes, as well as size nomenclature that lacks specific body measurements (e.g., ‘women’s size 6’). Additionally, manufacturers often do not produce clothing in all sizes, further limiting consumer choice. This practice is part of their ‘branding’ strategy, targeting their offerings to a specific subset of body types. Standardized sizing systems are based on median body measurements within a target demographic, with guidelines like ISO 8559 ensuring consistency across manufacturers and markets.

Traditional pattern pieces placed on a marker are cut out separately, surrounded by negative space. This separation often results in pieces not sharing borders, leading to unusable offcuts. While effective for mass production, traditional methods often result in significant fabric waste. Fletcher comments that the current approach and mindset accept these losses as an inevitable and acceptable part of the supply chain. Patterns are not always optimized for fabric efficiency, and the rigidity of traditional pattern making limits creativity and innovation, making it difficult to adapt to new sustainable practices, particularly on industrial scales.


\section{Zero Waste Pattern Making}
\subsection{Advent and Principles}
\subsection{Examples}
\subsection{Limitations}
Zero waste garments have a loose fit nature with greater eases around the body. Ease refers to the difference between body measurements and garment measurements, and it significantly affects fit, comfort, and style. There are two types of ease: wearing ease and design ease. Wearing ease is the minimum amount of extra fabric needed to allow comfortable movement, while design ease involves additional fabric for aesthetic purposes, such as creating a relaxed or oversized fit. Zero waste designs typically incorporate larger design ease to achieve their intended loose-fit aesthetic.

\section{Bespoke Clothing}
Fit and style are critical factors that consumers consider when purchasing a garment. Achieving the right fit is essential for consumer satisfaction and can prolong the life of the garment, thereby reducing waste. Customization refers to altering a garment to meet an individual’s specific physical dimensions and requirements. Personalization involves tailoring a garment to reflect the individual’s unique style preferences and tastes. Together, customization and personalization create bespoke clothing uniquely tailored to fit the individual’s body and style. This approach offers a more sustainable and consumer-centric alternative to mass-produced, ill-fitting fashion.

Furthermore, bespoke clothing fosters a deeper connection between the garment and the individual. Because each piece is meticulously tailored to fit personal measurements and style preferences, it holds greater significance for the wearer. This personal investment in the garment increases its perceived utility and value, encouraging individuals to wear and keep it for longer periods. Unlike fast fashion, which emphasizes quick turnover and low cost, bespoke clothing’s personalized nature leads to more sustained use and reduces the need for frequent replacements, thereby lowering the overall environmental impact.

The concept of a customized zero waste garment presents an intriguing contradiction, as it involves balancing fit and waste. Both contribute to eco-conscious consumerism in different ways. While zero waste designs aim to minimize fabric waste, bespoke garments emphasize perfect fit and personalization. Exploring this tradeoff between fit and waste is essential for developing more sustainable and personalized fashion solutions. This balance can lead to innovative approaches that reconcile the goals of zero waste design with the demand for bespoke tailoring, ultimately contributing to more environmentally friendly and consumer-focused garment production.

\section{Digitisation Trends in Fashion}
\subsection{Body Scanning and Digital Pattern Making}
\subsection{3D Modelling and Virtual Prototyping}
\subsection{Exisitng Computational Frameworks}




