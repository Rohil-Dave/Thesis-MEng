\chapter{Results}

\section{Workshop Study}
The workshop study had eight participants a mean maximum bodice circumferemce of 99.96 cm, median of 98.50 cm and a standard deviation of 8.00 cm. The mean shirt length is 60.36 cm, median of 59.70 cm, and a standard deviation of 6.99 cm. Six of eight completed the entire study, reaching finished garment stage and fit evaluation. Their garments are seen in Figure \ref{fig:workshop_garments}
\begin{figure}[htb]
    \centering
    \begin{subfigure}[b]{0.45\textwidth}
        \centering
        \includegraphics[width=\textwidth]{Images/Workshop_max_circ_Boxplot.png}
        \caption{}
    \end{subfigure}
    \hfill
    \begin{subfigure}[b]{0.45\textwidth}
        \centering
        \includegraphics[width=\textwidth]{Images/Workshop_desired_shirt_length_Boxplot.png}
        \caption{}
    \end{subfigure}
    \caption{}
\end{figure}
\newpage
\begin{figure} [H]
    \centering
    \includegraphics[width = \textwidth]{Images/workshop garments.png}
    \caption{Workshop study garments}
    \label{fig:workshop_garments}
\end{figure}

\begin{figure} [H]
    \centering
    \includegraphics[width = 0.8\textwidth]{Images/fit likert stacked bar.png}
    \caption{Fit Likert Scales}
\end{figure}
\begin{figure} [H]
    \centering
    \includegraphics[width = 0.8\textwidth]{Images/comfort likert stacked bar.png}
    \caption{Comfort Likert Scales}
\end{figure}


\begin{figure}[H]
    \centering
    \begin{subfigure}[b]{0.45\textwidth}
        \centering
        \includegraphics[width=\textwidth]{Images/Workshop_efficiency_ideal_Boxplot.png}
        \caption{}
    \end{subfigure}
    \hfill
    \begin{subfigure}[b]{0.45\textwidth}
        \centering
        \includegraphics[width=\textwidth]{Images/Workshop_efficiency_used_Boxplot.png}
        \caption{}
    \end{subfigure}
    \caption{}
\end{figure}

\begin{figure} [H]
    \centering
    \includegraphics[width = 0.5\textwidth]{Images/Workshop_CutLossArea_Bar.png}
    \caption{Workshop Cut Loss Area}
\end{figure}
\begin{figure} [H]
    \centering
    \includegraphics[width = 0.5\textwidth]{Images/Workshop_CutLossWidth_Bar.png}
    \caption{Workshop Cut Loss Width}
\end{figure}
\begin{figure} [H]
    \centering
    \includegraphics[width = \textwidth]{Images/Workshop_Eff_bar.png}
    \caption{Workshop Fabric Efficiency}
\end{figure}
\begin{figure} [H]
    \centering
    \includegraphics[width = \textwidth]{Images/Workshop_BoltWidths_Scatter.png}
    \caption{Workshop Ideal Bolts}
\end{figure}

Efficiency, bolt width, cut loss width, and cut loss area for Workshop Study comparing ideal bolt and actual used bolt. All efficiencies when using ideal bolt width are greater than 95\%.


\section{Body Scans Study}

The median is a measure of central tendency that is less affected by extreme values than the mean so should use that when many outliers present ???

\subsection{Distibutions}
Body Scans Study have a mean maximum bodice circumferemce of 99.58 cm, median of 98.25 cm and a standard deviation of 8.61 cm. The mean and median are close, indicating a relatively symmetrical distribution with a slight skew towards larger values. There is moderate variability (standard deviation of 8.613), with some outliers above 110 cm.

The mean shirt length is 62.40 cm, median of 64.5 cm, and a standard deviation of 6.72 cm. 

\begin{figure}[htb]
    \centering
    \begin{subfigure}[b]{0.45\textwidth}
        \centering
        \includegraphics[width=\textwidth]{Images/Mendeley_max_circ_Hist.png}
        \caption{Histogram}
    \end{subfigure}
    \hfill
    \begin{subfigure}[b]{0.45\textwidth}
        \centering
        \includegraphics[width=\textwidth]{Images/Mendeley_max_circ_Boxplot.png}
        \caption{Boxplot}
    \end{subfigure}
    \caption{Distribution of max bodice circumference for Body Scans Study}
\end{figure}

\begin{figure}[htb]
    \centering
    \begin{subfigure}[b]{0.45\textwidth}
        \centering
        \includegraphics[width=\textwidth]{Images/Mendeley_shirt_length_Boxplot.png}
        \caption{Histogram}
    \end{subfigure}
    \hfill
    \begin{subfigure}[b]{0.45\textwidth}
        \centering
        \includegraphics[width=\textwidth]{Images/Mendeley_shirt_length_Hist.png}
        \caption{Boxplot}
    \end{subfigure}
    \caption{Distribution of calculated shirt length for Body Scans Study}
\end{figure}

\subsubsection{Fabric Use}
Efficiency, bolt width, cut loss width, and cut loss area for 100 scans study comparing ideal bolt and hypothetical used bolt.
When using the ideal bolt width, the mean efficiency is 98\%, median efficiency is 98\%, and the standard deviation is 1\%. The mean ideal bolt width is 133 cm and median ideal bolt is 130 cm.
\begin{figure} [H]
    \centering
    \includegraphics[width = \textwidth]{Images/Mendeley_Plot.png} 
    \caption{Used and Ideal Efficiency Metrics for 100 Scans Study}
\end{figure}

Cohort analysis
\begin{table} [H]
    \centering
    \begin{tabular}{p{2cm}|p{2cm}|p{2cm}|p{2cm}|p{2cm}|p{2cm}|p{2cm}}
        \textbf{Bolt Width} & \textbf{Mean Eff} & \textbf{Median Eff} & \textbf{St Dev Eff} & \textbf{Mean Loss Area} & \textbf{Median Loss Area} & \textbf{St Dev Loss Area}\\
        \hline % this produces a horizontal line, this could be used elsewhere in the table
        110& 81.7\% & 83.6\% & 6.1\% & 2620.64 & 2318.0 & 889.32\\
        115& 78.5\% & 80.0\% & 6.0\% & 3233.87 & 2922.75 & 937.46\\
        120& 76.2\% & 77.1\% & 7.4\% & 3737.44 & 3556.88 & 1207.17\\
        125& 78.7\% & 75.6\% & 12.1\% & 3526.84 & 4025.0 & 2097.15\\
        130& 83.7\% & 91.7\% & 13.5\% & 2771.99 & 934.5 & 2542.75\\
        135& 87.9\% & 92.9\% & 12.3\% & 1997.82 & 843.0 & 2501.29\\
        140& 88.8\% & 91.1\% & 9.2\% & 1716.44 & 1131.0 & 2025.64\\
        145& 87.3\% & 88.1\% & 8.1\% & 1865.09 & 1491.63 & 1849.15\\
        150& 85.2\% & 85.5\% & 7.1\% & 2158.84 & 1890.75 & 1679.22\\
        155& 83.3\% & 83.1\% & 6.2\% & 2418.42 & 2333.63 & 1388.62\\
        160& 81.2\% & 80.6\% & 5.4\% & 2752.84 & 2795.13 & 1076.74\\
        165& 79.1\% & 78.3\% & 5.2\% & 3089.52 & 3250.13 & 782.20\\
        \end{tabular}
    \caption{Description of Cohort Analysis on Different Bolts for 100 Scans Study}
\end{table}
\textbf{SHOULD CHANGE FORMAT OF GRAPHS BELOW BECAUSE NOT CONTINOUS???}
\begin{figure}[H]
    \centering
    \begin{subfigure}[b]{0.45\textwidth}
        \centering
        \includegraphics[width=\textwidth]{Images/Cut Loss Mean for Different Bolt Widths for 100 scans.png}
        \caption{Cut Loss Area}
    \end{subfigure}
    \hfill
    \begin{subfigure}[b]{0.45\textwidth}
        \centering
        \includegraphics[width=\textwidth]{Images/Eff Mean for Different Bolt Widths for 100 scans.png}
        \caption{Efficiency}
    \end{subfigure}
    \caption{Mean Cut Loss Area Used and Efficiency Used for Different Bolt Widths for 100 Scans Study}
\end{figure}

For this cohort, using a 140 cm wide fabric bolt minimises the cut loss area and maximises the efficiency.

Count of pattern orientation on starting fabric and possibility of embellishment potential for various bolt widths ranging from 110 to 160 cm.
\begin{figure} [H]
    \centering
    \includegraphics[width = \textwidth]{Images/Mendeley_Bar.png}
    \caption{Layout and Embellishment Possibility for 100 Scans Study}
\end{figure}
For the range of bolt widths between 110 and 160 cm, each pattern generated for the Mendeley dataset participants will fit in the regular or rotated orientation. No mitigation strategies need to be employed within this range of bolts.

\textbf{ADD recoup of cut loss area due to embellishment}

\section{Personal Case Study}
