\chapter*{Abstract}

The fashion industry is a significant contributor to environmental degradation, second only to oil and mining. Waste and inefficiencies are endemic over the entire lifecycle of a garment from yarn and fabric manufacture to clothing production to final disposal. According to the Ellen MacArthur Foundation (2017), the industry is responsible for approximately 10\% of global carbon emissions and is the second-largest consumer of the world’s water supply.

This report focuses on reducing waste during garment manufacturing, where \~15\% of all fabric is wasted. Designs with increased garment utility and desirability are explored. ‘Zero waste patterns’ are one such class of designs. This study looks at parameterisation of these patterns based on individual body measurements and suggests mitigation techniques like embellishments. New tools that allow end users to visualize drape and fit of the 2D patterns in 3D are investigated.

These parameterisation heuristics were tested in a workshop setting and participants’ responses to ease of tailoring as well as suitability and fit of the design are collated. Lastly, these patterns were tested on a publicly available data set of scanned body measurement and efficiency (minimized fabric loss) were computed for industry standard bolt widths. 
