\chapter*{Abstract}

This report focuses on reducing waste during garment manufacturing, where about 15\% of all fabric is wasted. Designs with increased garment utility and desirability are explored. ‘Zero waste patterns’ are one such class of designs. This study looks at parameterisation of these patterns based on individual body measurements and suggests mitigation techniques like embellishments. New tools that allow users to visualise drape and fit of the 2D patterns in 3D were investigated.
These parameterisations were tested in a workshop setting and participants' responses to ease of tailoring as well as suitability and fit of the design were collated. Parameterisation was also tested on a publicly available dataset of scanned body measurement and efficiencies (minimised fabric loss) were computed for industry standard bolt widths. A mean efficiency metric of 98.1\% was achieved for ideal bolt width.
Mixed mode analysis was also performed on the author's scan data to produce an actual garment and the pattern was tested virtually in CLO 3D.
These results show that it is possible to use `tail0r’ to parametrise a zero-waste pattern to fit individual body measurements and create bespoke garments with reduced fabric cut losses.